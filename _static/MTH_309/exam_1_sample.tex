\documentclass[14pt]{article}

\usepackage[margin=1.5cm]{geometry}
\usepackage{enumerate}
\usepackage{amsmath,amssymb}
\usepackage{tikz}

\begin{document}

\begin{center}
Sample Exam 1 - Systems of Equations
\end{center}

You will have 30 minutes to complete the exam.  You may use a calculator, but you must show all steps done to get full credit for completing the problem.  This means that if you use your calculator for anything other than arithmetic, you must indicate on your test paper what you did on the calculator.

\begin{enumerate}
\item 
%A company manufactures three products, call them A, B, and C.  Product A has a cost of \$50, requires 7 units of materials to make, and can be sold for \$75.  Product B has a cost of \$25, requires 10 units of materials, and can be sold for \$40.  Product C has a cost of \$40, requires 15 units of materials, and can be sold for \$100.  You have a total budget of \$1000, 270 units of materials on hand, and a revenue target of \$1800. Write, but {\bf do not solve}, a system of linear equations, in standard form, that can be solved to find the number of each product to make.
%
%
Consider a degree 4 polynomial $ p(x) = p_0 + p_1x + p_2x^2 + p_3x^3 + p_4x^4 $ which passes through the points (-2,-1), (-1,-8), (0,0), (1,4), and (2,0).  Write, but {\bf do not solve}, a system of linear equations, in standard form, that can be solved to find the coefficients of $ p(x) $.
%
%
%Consider the chemical reaction below.  Write, but {\bf do not solve}, a system of linear equations, in standard form, that can be solved to find the reaction coefficients.
%\[
%C_4H_8NH + O_2 \longrightarrow CO_2 + H_2O + NH_3
%\]
%
%
%Consider the ystem of 1-way roads shown below.  Write, but {\bf do not solve}, a system of linear equations, in standard form, that models the traffic flows.
%\begin{center}
%\begin{tikzpicture}[scale=2]
%\node at (0,0) {1};
%\node at (0,2) {2};
%%\node at (1,1) {5};
%\node at (2,0) {4};
%\node at (2,2) {3};
%
%\draw (0,0) circle (0.2cm);
%\draw (0,2) circle (0.2cm);
%%\draw (1,1) circle (0.3cm);
%\draw (2,0) circle (0.2cm);
%\draw (2,2) circle (0.2cm);
%
%\draw[->,thick] (-1,0) -- (-0.3,0);
%\draw[->,thick] (0,0.3) -- (0,1.7);
%\draw[->,thick] (0.3,2) -- (1.7,2);
%\draw[->,thick] (2,1.7) -- (2,0.3);
%\draw[->,thick] (1.7,0) -- (0.3,0);
%\draw[->,thick] (-0.3,2) -- (-1,2);
%\draw[->,thick] (2.3,0) -- (3,0);
%\draw[->,thick] (3,2) -- (2.3,2);
%%\draw[->,thick] (0.21,0.21) -- (1.79,1.79);
%\draw[->,thick] (1.79,0.21) -- (0.21,1.79);
%%\draw[->,thick] (0.21,0.21) -- (0.79,0.79);
%%\draw[->,thick] (0.79,1.21) -- (0.21,1.79);
%%\draw[->,thick] (1.79,0.21) -- (1.21,0.79);
%%\draw[->,thick] (1.21,1.21) -- (1.79,1.79);
%
%\node at (-0.6,0.1) {$x_1$};
%\node at (-0.6,2.1) {150};
%\node at (2.6,0.1) {$x_4$};
%\node at (2.6,2.1) {50};
%\node at (1,0.1) {$10$};
%\node at (1,2.1) {$x_5$};
%\node at (-0.1,1) {$x_2$};
%\node at (2.1,1) {$x_3$};
%\node at (1.1,1.1) {30};
%%\node at (1.35,1.5) {20};
%%\node at (1.6,0.55) {$x_5$};
%%\node at (0.4,0.55) {$x_6$};
%\end{tikzpicture}
%\end{center}

\item Solve the system of equations below.  Make sure at each step you indicate what you are doing.
\begin{align*}
2x+2y+4z &= 10 \\
2x-y+z &= 1 \\
-3x+y+3z &= -6
\end{align*}

\item (TRUE or FALSE)  Consider the following statement and decide if it is true or false.  If it is true, provide reasoning.  If it is false, provide a counterexample.  
\begin{center}
``If the reduced row-echelon form of the {\bf coefficient matrix} has a pivot in every column, then the system has a unique solution."
\end{center}

\item How many solutions does the following system of equations have?  Be sure to back up your answer with an explanation.
\begin{align*}
2x +12y -6z &= 6 \\
y -5z &= -7 \\
-3x -22y +29z &= -15
\end{align*}
%\item Consider the following system matrix in reduced row echelon form. How many solutions does the system have?
%\[
%A = \left[\begin{array}{rrr|r} 1 & 2 & 0 & -1 \\ 0 & 0 & 0 & 0 \\ 0 & 0 & 1 & -4 \end{array}\right].
%\]

%\item Reduce the following matrix to row echelon form:
%\[
%A = \left[\begin{array}{rrrr} 1 & 2 & 0 & -1 \\ 3 & 0 & -4 & -5 \\ -1 & -1 & 1 & 0 \end{array}\right].
%\]

\end{enumerate}

\end{document}